\documentclass{article}

% if you need to pass options to natbib, use, e.g.:
%     \PassOptionsToPackage{numbers, compress}{natbib}
% before loading neurips_2019

% ready for submission
% \usepackage{neurips_2019}

% to compile a preprint version, e.g., for submission to arXiv, add add the
% [preprint] option:
\usepackage[preprint]{neurips_2019}

% to compile a camera-ready version, add the [final] option, e.g.:
%\usepackage[]{neurips_2019}

% to avoid loading the natbib package, add option nonatbib:
%     \usepackage[nonatbib]{neurips_2019}

\usepackage[utf8]{inputenc} % allow utf-8 input
\usepackage[T1]{fontenc}    % use 8-bit T1 fonts
\usepackage{hyperref}       % hyperlinks
\usepackage{url}            % simple URL typesetting
\usepackage{booktabs}       % professional-quality tables
\usepackage{amsfonts}       % blackboard math symbols
\usepackage{nicefrac}       % compact symbols for 1/2, etc.
\usepackage{microtype}      % microtypography

\usepackage[pdftex]{graphicx}

\title{Empirical Privacy Loss Examples}

% The \author macro works with any number of authors. There are two commands
% used to separate the names and addresses of multiple authors: \And and \AND.
%
% Using \And between authors leaves it to LaTeX to determine where to break the
% lines. Using \AND forces a line break at that point. So, if LaTeX puts 3 of 4
% authors names on the first line, and the last on the second line, try using
% \AND instead of \And before the third author name.

\author{%
  Abraham D.~Flaxman \\
  Department of Health Metrics Sciences\\
  University of Washington\\
  Seattle, WA, USA \\
  \texttt{abie@uw.edu} \\
  % examples of more authors
  \And
  Sam Petti? \\
  % Affiliation \\
  % Address \\
  % \texttt{email} \\
  % \AND
  % Coauthor \\
  % Affiliation \\
  % Address \\
  % \texttt{email} \\
  % \And
  % Coauthor \\
  % Affiliation \\
  % Address \\
  % \texttt{email} \\
  % \And
  % Coauthor \\
  % Affiliation \\
  % Address \\
  % \texttt{email} \\
}

\begin{document}

\maketitle

\begin{abstract}
  The 2020 US Census will use $\epsilon$-Differential Privacy, and use the TopDown mechanism to guarantee privacy loss of at most $\epsilon$.  However, it is possible that there is some slack in the bounds, and in practice, the privacy loss will be substantially less than $\epsilon$.  We developed an empirical measure of privacy loss, and applied it to a range of examples inspired by some aspects of the TopDown mechanism, to better understand how the empirical privacy loss of census-style count queries might compare to the theoretical guarantee.
\end{abstract}

\section{Introduction}

The 2020 US census will use $\epsilon$-Differential Privacy.

The TopDown algorithm applies the geometric mechanism repeatedly, at multiple levels of a geographic hierarchy, using optimization to combine the noisy measurements and ensure consistency.

The additive property of DP ensures that the total privacy loss of TopDown is at most $\epsilon$, but it is possible that this inequality is not tight.

We investigated the privacy loss of an idealized top-down mechanism, using a nonparametric approach to estimating the empirical privacy loss.

\section{Methods}
\label{methods}

DP definition, and specialization to count queries, and to really simple count queries.

Optimization applied to results of multiple count queries is at the root of the question here.

\subsection{Simulation strategy for generating DP count data}

\subsection{Estimation of empirical privacy loss}

\section{Results}
\label{results}

Use natbib command \citet{hasselmo} to get citations, such as  Hasselmo, et al.\ (1995)


\begin{figure}
  \centering
   \includegraphics[width=0.8\linewidth]{myfile.pdf}
  \caption{Sample figure caption.}
\end{figure}

\section{Discussion}


\section*{References}

[1] Alexander, J.A.\ \& Mozer, M.C.\ (1995) Template-based algorithms for
connectionist rule extraction. In G.\ Tesauro, D.S.\ Touretzky and T.K.\ Leen
(eds.), {\it Advances in Neural Information Processing Systems 7},
pp.\ 609--616. Cambridge, MA: MIT Press.

[2] Bower, J.M.\ \& Beeman, D.\ (1995) {\it The Book of GENESIS: Exploring
  Realistic Neural Models with the GEneral NEural SImulation System.}  New York:
TELOS/Springer--Verlag.

[3] Hasselmo, M.E., Schnell, E.\ \& Barkai, E.\ (1995) Dynamics of learning and
recall at excitatory recurrent synapses and cholinergic modulation in rat
hippocampal region CA3. {\it Journal of Neuroscience} {\bf 15}(7):5249-5262.

\end{document}
